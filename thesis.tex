\documentclass[twoside]{uva-inf-bachelor-thesis}
\usepackage[english]{babel}
\usepackage{outlines}
\usepackage{amsmath}
\usepackage{csquotes}
\usepackage{biblatex}
\usepackage{braket}
\usepackage{xspace}
\usepackage{booktabs}
\usepackage{listings}

\DeclareMathOperator{\lcm}{lcm}
\newcommand{\code}[1]{\lstinline[breaklines=true]{#1}}
\newcommand{\ucosiii}{\textmu C/OS-III\xspace}
\newcommand{\ucosii}{\textmu C/OS-II\xspace}
\newcommand{\ucpu}{\textmu C/CPU\xspace}
\newcommand{\ucos}{\textmu C/OS\xspace}
\newcommand{\task}[1]{\ensuremath{\tau_#1}}

\addbibresource{refs.bib}
\usepackage{graphicx}

\usepackage{fontspec}
%\setmainfont[Ligatures={Common,TeX}]{Adobe Caslon Pro}
%\setsansfont{Source Sans Pro}
%\setmonofont{Iosevka}

\lstset{language=[ARM]Assembler,numbers=left,basicstyle=\small\ttfamily}


\makeatletter
\define@key{Gin}{resolution}{\pdfimageresolution=#1\relax}
\makeatother

\title{Predictable, guarantee-backed\\dynamic priority scheduling in the \ucosiii real-time operating system}
\author{Sam van Kampen}
\supervisors{T. Walstra}
\signedby{Signees}

\includeonly{scheduling}

\begin{document}

\maketitle

\begin{abstract}
    In this thesis, the performance and utility of Earliest Deadline First scheduling in combination with the processor demand guarantee criterion is evaluated in the \ucosiii real-time operating system on a first-generation Raspberry Pi, and contrasted with \ucosiii's default, priority-based scheduler, using the Rate Monotonic algorithm. The EDF scheduler was found to perform significantly better at little extra implementation cost, and the processor demand criterion was found to accurately determine schedulability in practice.
\end{abstract}


\tableofcontents

%
%
% I N T R O D U C T I O N
%
%

\include{introduction}

%
%
% R E L A T E D  W O R K
%
%

\include{related_work}

%
%
% S C H E D U L I N G
%
%

\include{scheduling}

%
%
% H A R D W A R E
%
%

\chapter{Hardware} \label{chp:hardware}
Real-time operating systems run on a broad range of hardware. Running tests on all of this different hardware is, of course, impossible. For this thesis, I have chosen to use a piece of hardware which exemplifies a number of characteristics of real-time systems, but which is also readily available and well-documented: a first-generation Raspberry Pi.

\section{The Raspberry Pi 1B}
The hardware used in this thesis is a Raspberry Pi 1B, a single-board computer from early 2012. While it was marketed as an educational `toy' for children to learn how to program on, it became popular mainly as a cheap development board for (hardware) hobbyists, due to its ability to control electronics using its general purpose I/O (GPIO) pins, and its low price point of \$25-35 (depending on model). Its hardware is as follows:

\begin{outline}
    \1 A Broadcom BCM2835 System-on-Chip, including:
        \2 A single-core 700MHz ARM11 (ARMv6) processor
        \2 A VideoCore IV graphics processor
        \2 512 megabytes of RAM
    \1 26-pin General Purpose I/O (GPIO) header, capable of performing various functions
        \2 Serial I/O, SPI, software-controlled reading and writing at 3.3V
    \1 Ethernet, USB, HDMI and composite out, et cetera.
\end{outline}

The Pi has a number of desirable characteristics that make it suitable for use in this thesis. Firstly, it uses a low-power ARM processor, an architecture which is very common in embedded systems (with a market share of 37\% at the end of 2014 \cite{arm:embeddedmarketshare}). Additionally, its GPIO pins and their support for serial I/O allow for a simple way of interacting with the Raspberry Pi without having to implement USB or ethernet drivers. Such serial connections are also very common in embedded systems. Lastly, due to the popularity of the Pi, it is fairly well-documented. The datasheet that describes the hardware in the Raspberry Pi and how to interface with it is freely available\cite{bcm:2835peripherals}, and although it is occasionally inaccurate, there is a thorough list of errata available online\cite{bcm:2835errata}. Where the datasheet has omissions, there is also often information available online.

\subsection{General Purpose I/O}
As described above, the Raspberry Pi has 26 general purpose I/O pins. Some of these pins have a fixed function, but many have multiple functions, and which one a pin uses can be controlled by software. An overview of the 26-pin header and the functions of its pins can be seen in figure~\ref{fig:gpiopinout}. One thing of note is the fact that the pin numbering used on the header is not the same as is used in the BCM2835 peripherals manual; for instance, pin 7 on the header is GPIO pin 4 in the data sheet.

\begin{figure}[ht]
    \centering
    \includegraphics[scale=0.5]{figures/Pi-GPIO-header-26-sm.png}
    \caption{The Raspberry Pi 1B GPIO pinout. Pins labeled \textit{GPIO \#} correspond to GPIO pins in the Broadcom BCM2835 peripherals manual\cite{bcm:2835peripherals}.}
    \label{fig:gpiopinout}
\end{figure}

\subsection{The VideoCore IV processor}
Coming from the x86 world, you might expect the VideoCore IV (VC4) GPU to be little more than a standard PC graphics card, managing and accelerating graphical output. While it does perform those actions, the VC4 has much broader responsibilities on the Raspberry Pi. It runs a (real-time) operating system of its own, ThreadX\cite{rpi:opensourcevpu}, and handles system initialization and the early boot process\cite{rpi:bootforum}. It also performs power and clock management\cite{rpi:gpuclockpower}. The way the clock management functions is however conveniently absent from the BCM2835 datasheet.



%
%
% P O R T I N G
%
%

\chapter{Porting \ucosiii to the Raspberry Pi}
To run \ucosiii on the Raspberry Pi, the operating system needs to be adapted to run on the hardware the Pi uses. These adaptations are commonly called \textit{ports}. There are ports to architectures that are similar to that of the Pi, such as the ARM9-based NXP LPC2923 \cite{micrium:nxplpc}, but there is no port for the Broadcom BCM2835, the SoC used by the Pi. This work includes such a port.

\section{The structure of \ucosiii}
The majority of \ucosiii is written as processor-independent C code, and can therefore be re-used in the port verbatim. Fragments of the operating system whose implementation is highly dependent on the target hardware, such as task switching or interaction with hardware timers, are implemented in separate source files. There is a CPU-specific component, \ucpu, which specifies CPU features such as the availability of a count-leading-zeros instruction, endianness, the size of data types such as timestamps, addresses, et cetera. Separately, OS-specific components such as task switching functions are defined. Lastly, a `Board Support Package' is defined, which contains the functionality that interacts with the on-board components such as the hardware timer and handles interrupts.

Of particular interest to the scheduler are the timer and the interrupt handling code.

\section{Bootup}
As mentioned in the section on hardware, the Raspberry Pi's startup process is mostly handled by the VideoCore IV GPU. The boot process, as described in \textcite{rpi:bootforum}, is roughly as follows. The first-stage bootloader is programmed into ROM when the Raspberry Pi is manufactured, and it loads the second-stage bootloader \code{bootcode.bin} from the first SD card partition. \code{bootcode.bin} loads the GPU firmware, \code{start.elf}, which initializes hardware and sets up memory based on parameters in \code{config.txt}, loads the kernel (by default \code{kernel.img}) to address \code{0x8000}, and releases the ARM CPU from reset.


The port startup code (\code{startup.s}) subsequently configures and initializes the processor, by setting up stack pointers, initializing interrupt vectors and enabling the floating-point unit, branch predictor, and instruction cache\footnote{Data cache cannot be enabled without using virtual memory, as it would interfere with memory-mapped peripherals. It is left disabled in the interest of time.}. After initializing the C environment by zeroing out the BSS, control is transferred to the C main function. In C, hardware such as serial I/O ports and the timer interrupt are configured using functions in the board support package. Lastly, the OS is initialized with the \code{OSInit} function, tasks are created, and the operating system is started with the \code{OSStart} function.

\section{Timers}
In real-time systems, there obviously needs to be a correspondence between the time as measured in the system and external time. This is where hardware timers come in. The Raspberry Pi provides both running counter functionality (which allows the system to have a clock -- though it is not a real-time clock, which would keep an accurate date and time of day across reboots) and timer functionality (so the system is interrupted after a given time is elapsed).

As described in the BCM2835 peripheral data sheet, the Raspberry Pi contains two timers; a timer for the ARM processor itself and a system timer. Although \textcite{sfd:realpi} opts to use the ARM timer, the peripheral data sheet suggests using the system timer when accurate timing is required.\cite[p. 196]{bcm:2835peripherals}.

The system timer has microsecond resolution\footnote{Although the ARM timer has a higher frequency, it is not useful to us as the time taken to switch between tasks is already in the microsecond range.}, and consists of a 64-bit running counter and four 32-bit timer channels. These timer channels have a corresponding match register, which contains a 32-bit value that is compared with the running counter, triggering an interrupt when the lowest 32 bits of the running counter match the register value.

Two of the timer channels are in use by the VideoCore IV GPU, and are therefore not usable. Timer channel 1 and 3 are, however, available. We only need a single timer channel, and currently use timer channel 1.

\section{Interrupts}
As discussed in the previous section, the hardware timer can trigger an interrupt when a timer channel is matched. The system timer is one of the hardware components which is poorly described in the peripheral data sheet, but information found online can close that gap.

Interrupt handling in the ARM architecture differs significantly from similar features in the x86 architecture. Where the latter uses an interrupt vector table which dispatches a specific interrupt to a specific interrupt service routine, the former uses a much more generic approach.

ARM uses the term `exception' to mean an event which causes the processor to stop execution and jump to a piece of code to handle it. The ARM architecture supports seven kinds of exceptions: \textit{Reset}, \textit{Undefined Instruction}, \textit{Software Interrupt}, \textit{Prefetch Abort}, \textit{Data Abort}, \textit{Interrupt} (IRQ) and \textit{Fast Interrupt} (FIQ). When an exception is generated, the CPU jumps to an \textit{exception vector} for the given exception. These vectors are usually located at the beginning of the address space, but can be remapped. An overview of exception types, their corresponding processor mode and address that execution starts at after exception reception is given in table~\ref{tbl:exceptions}.

\begin{table}[h]
    \centering
    \begin{tabular}{lll}
        \toprule
        \textbf{Exception type} & \textbf{Processor mode} & \textbf{Execution address} \\
        \midrule
        Reset & Supervisor & \texttt{0x00000000} \\
        Undefined instructions & Undefined & \texttt{0x00000004} \\
        Software interrupt & Supervisor & \texttt{0x00000008} \\
        Prefetch Abort & Abort & \texttt{0x0000000C} \\
        Data Abort & Abort & \texttt{0x00000010} \\
        IRQ (interrupt) & IRQ & \texttt{0x00000018} \\
        FIQ (fast interrupt) & FIQ & \texttt{0x0000001C} \\
        \bottomrule
    \end{tabular}
    \caption{An overview of ARM exceptions. Adapted from table A2-4 in the ARM Architecture Reference Manual\cite{arm:arm}.}
    \label{tbl:exceptions}
\end{table}

In effect, processor faults and software interrupts have their own exception, and all `normal' device interrupts are coalesced into the `interrupt' and `fast interrupt' exceptions. The exception handling code, then, is tasked with determining which device caused the interrupt. The means of determining the interrupting device differ, as they often rely on hardware-provided memory-mapped registers. This is no different on the Raspberry Pi -- the peripheral data sheet details the interrupt mechanism in some detail on page 109. The peripheral data sheet however fails to mention the interrupt number assigned to the system timer. Some sleuthing reveals that the system timer channels are mapped at the start of the interrupt numbers, so IRQ 0 refers to the first timer channel, IRQ 1 to the second, et cetera.

As mentioned, the interrupts are handled either by the `fast interrupt' handler or the normal `interrupt' handler. The BCM2835 allows the systems programmer to route a single interrupt source to the fast interrupt vector, which has more banked registers and therefore allows for faster interrupt processing. This is not used in the port currently, as it would complicate the task switching code, which can currently use the same assembly for returning to voluntarily yielded code and interrupted code.

\section{Serial I/O} \label{sec:miniuart}
Implementing input/output capabilities is not strictly required to get \ucosiii running on the Raspberry Pi. An operating system is of little use without any I/O capabilities, however, and being able to output debugging information is also of great use in porting.

The type of I/O that the port has support for is very common in embedded devices, and was used throughout much of the twentieth century for communication between computers and associated terminals: serial communication using a UART. The peripheral data sheet tells us that the Raspberry Pi contains two UARTs, a primary ARM PL011 UART and a secondary so-called `mini UART'. Due to limitations such as the shallow FIFOs, this port uses the primary UART.

On the hardware side, the UART uses GPIO pins 8 and 10\footnote{Pin 14 and 15, respectively, in the Broadcom peripheral datasheet.} for transmit and receive, respectively. Additionally, one of the Pi's ground pins needs to be used to ensure the communicating devices have a common ground.



%
%
% I M P L E M E N T A T I O N
%
%

\include{implementation}

%
%
% E X P E R I M E N T S
%
%

\chapter{Experiments}
\section{Task switch time}
The time taken to switch between tasks is evaluated, for both round-robin task switching and normal inter-priority task switching.

\subsection{\ucosiii scheduler: round-robin task switch time}
The round-robin task switch time under the default \ucosiii scheduler is evaluated (code in: \code{apps/rrmeasure/app.c}). The tick rate was set to 100Hz. The task switch time was sampled 1,024,000 times, and a histogram of the task switch time can be seen in figure~\ref{fig:rrhist}. The measured time is the time to yield a given task and switch in another (i.e. is measured from just before a yield call until control is given to the next task).

A clear primary peak is visible around 8µs. However, larger task switch times do occur. One explanation for them lies in the tick interrupt, which suspends the task for some time, thus resulting in a larger task switch time. Another phenomenon, whose cause is not entirely clear, has something to do with the number of round-robin tasks:  every $n$ task switches after some number of ticks, an outlier is produced.

\begin{figure}[htpb]
    \centering
    \includegraphics[width=\textwidth]{figures/task_switch_time.eps}
    \caption{A histogram of task switch time between round-robin tasks (mean 8.17µs; std. dev. 0.90µs). Total samples: 1,024,000.}
    \label{fig:rrhist}
\end{figure}

\subsection{\ucosiii scheduler: task switch time as the number of priorities increases}
As the number of priorities increases, the size of the priority bitmap increases linearly. Since we search the priority bitmap each time we perform a task switch, we would expect task switch time to increase linearly as well. In figure~\ref{fig:prioboxplot}, a set of box plots of task switch time is shown, for a varying number of priorities. For each number of priorities, 32 tasks were spaced evenly in the priority space, and the time to switch between them was measured for each task switch for a duration of five seconds. As expected, the mean task switch time increases linearly (visible as a curved line in this log-log plot due to the non-zero y-intercept). Additionally, the variance increases greatly as the number of priorities increases.

\begin{figure}[htpb]
    \centering
    \includegraphics[width=\textwidth]{figures/boxplot.eps}
    \caption{Box plots of task switch time as the number of priorities increases. As task switches were measured over a fixed period of time, the number of samples for each box plot varies, from a minimum of 547,535 to a maximum of 938,926.}
    \label{fig:prioboxplot}
\end{figure}

\section{Task set schedulability}
In this experiment, the new EDF scheduler is compared to the built-in scheduler when it comes to task set schedulability. Random task sets were generated and EDF schedulability was analyzed using the processor demand criterion, then confirmed by running the task sets on hardware. The same task sets were run using the Rate Monotonic algorithm and the built-in fixed priority scheduler. The method by which task sets were generated and run is discussed below.

\subsection{Determining worst-case execution times}
As discussed in the chapter on scheduling, when using guarantee-backed scheduling, an accurate worst-case execution time estimation for tasks is paramount to ensure that the provided guarantees actually have meaning. The same is true when evaluating scheduling performance. However, worst-case execution time estimation is a difficult task for even simple code, and in these experiments, I would like to greatly vary task sets, requiring me to perform worst-case execution time analysis for many task sets. Therefore, I have chosen to simulate actual tasks by executing instructions which do not perform any useful work, but have very predictable timing characteristics. These timing characteristics can be found in chapter 16 of the ARM Technical Reference Manual for the Pi's ARM1176JZF-S processor\cite{arm:arm1176}.\\


\begin{lstlisting}
wait_for_cycles:
    lsr r0,r0,#1
loop$:
    subs r0, r0, #1
    bne loop$
    bx lr
\end{lstlisting}

\begin{enumerate}
    \item The \code{wait_for_cycles} subroutine waits for approximately $r_0$ cycles. A description of the cycle count of the used instructions is as follows;
    \item The \code{lsr} instruction is syntactic sugar for a \code{mov} instruction with included shift, and therefore takes a single cycle\cite[p. 16-7]{arm:arm1176}. It requires its source register as an early register, but that should be no issue as this is the first instruction in the function.
    \setcounter{enumi}{3}
    \item Data processing instructions not targeting the program counter and without included shifts, such as this \code{subs} instruction, take a single cycle\cite[p. 16-7]{arm:arm1176}.
    \item Branch instructions such as \code{bne} have complex timing characteristics, since the ARM1176JZF-S processor includes both a static and a dynamic branch predictor, as well as a return prediction stack\cite[p. 16-2]{arm:arm1176}. Let us assume a scenario where we want to wait for a number of cycles that is greater than one.

    The first time this branch executes and when it is not in the 128-entry dynamic branch predictor cache, it will take 4 cycles, as it will be correctly predicted to be taken by the static branch predictor, which predicts that backward branches are always taken\cite[p. 5-5]{arm:arm1176}.

    After this, it will be set to be weakly taken in the dynamic branch predictor. This correct prediction allows the branch to take a single cycle\footnote{If folded out, the branch could take zero cycles. Since branch folding is hard to predict, however, my startup code disables branch folding.}\cite{arm:arm1176}.

    After the last iteration, the branch will be mispredicted, and as the subtraction instruction directly precedes the branch instruction, this incurs a cost of 6 cycles.
    \item The return instruction takes 4 or 5 cycles depending on whether the code is interrupted - if it is, the return stack will most likely be empty and therefore cause an extra cycle.
\end{enumerate}

The cycle count for the given instructions could vary based on presence in the instruction cache or other architectural buffers (such as the instruction prefetch buffer). Most important, however, is the timing of the instructions in the inner loop, as they account for the bulk of the cycles used. Experimentally, the 2-cycle inner loop behavior is verified in \code{apps/waittest/app.c}.

One issue with using cycles to wait for a given period of time is the variability of the Pi's clock speed. The Pi will, however, only throttle the clock speed in two cases; when an undervoltage is detected (this occurs when the power supply cannot supply 5 volts at the required amperage, so the voltage drops) or when the core temperature gets above 80 degrees\cite{rpi:gpuclockpower}. Measuring the GPIO voltage during system stress reveals that my power supply can consistently supply five volts. The second case does not occur on first-generation Pis due to its low-power processor (later generations have higher clock speeds and are multi-core chips, and therefore have a larger thermal output and are at higher risk of overheating). Therefore, we can rely on the system clock staying constant.

\subsection{Generating task sets}
To measure task set schedulability, a range of random task sets was generated. The parameters for these task sets were as follows:

\begin{itemize}
    \item \textbf{Target processor utilization} varied between 65\% and 100\%, skewing slightly toward higher processor utilization since fewer randomly generated task sets hit the higher processor utilization. In the end, the accepted task set with the highest processor utilization had $U \approx 0.985$.
    \item The \textbf{task utilization} was picked uniformly between zero and the remaining available processor utilization
    \item The \textbf{period} was chosen between 250 and 1000 milliseconds, in multiples of 10 to get a tractable hyperperiod.
    \item The \textbf{worst-case execution time} of a task was derived by multiplying a task's utilization with its period.
    \item The \textbf{relative deadline} of a task was generated by subtracting a third of the difference between the period and the worst-case execution time.
    \item The \textbf{number of tasks} varied between 5 and 10, with a mean of 5.75 tasks in a task set.
\end{itemize}

\noindent Additionally, the \ucos tick task was added to the task sets when computing the scheduling guarantee. As noted in the chapter on implementation, parameters of the tick task were estimated as $T_{\text{tick}} = 1$ tick, $C_\text{tick} = 50$\textmu s , $D_\text{tick} = 250$\textmu s.

10,000 task sets were generated. Task sets that had fewer than 5 tasks or a hyperperiod that was longer than 100 seconds were excluded. This left 2407 task sets, which were partitioned into 1929 task sets that passed the schedulability guarantee and 478 task sets that did not. These task sets were subsequently run.

\subsection{Running task sets}
To validate the schedulability guarantee, the task sets were executed on the real hardware. To evaluate the practical schedulability of a task set, the task set was run until the end of its hyperperiod, or until its first deadline miss, whichever came first. Since, in all of these task sets, the relative deadlines of tasks are smaller or equal to their period, when the hyperperiod passes, the schedule repeats itself, and the task set does not need to be run any longer to evaluate schedulability.

The data collection was done automatically, by restarting the Raspberry Pi using its hardware watchdog after task set execution, compiling a new kernel with the next task set, and loading that over the serial connection. Execution was then monitored, marking a task set as `failed' when a deadline miss was reported, and as `successful' otherwise.

\subsubsection{Results using the EDF scheduler}
All 1929 task sets that passed the schedulability guarantee were run without any deadline misses. Of the 478 rejected task sets, 46 ran without any deadline misses, whereas the other 432 missed at least one deadline.

\subsubsection{Results using the RM scheduler}

\emph{to be filled in later}


%
% D I S C U S S I O N / C O N C L U S I O N
%

\include{discussion_conclusion}

{
    \hfuzz=8pt
    \printbibliography
}
\end{document}
